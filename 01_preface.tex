\chapter*{Preface}

These are the lecture notes on "Hamiltonian Systems with Constraints" given by Prof. Vechernin at Saint Petersburg State University which I typesetted in \LaTeX \ during my exchange semester in the winter term 2018/19. \\

The main reason for doing this was to enable the Russian students a better preparation for the exam because such typesetted notes didn't exist. Moreover, such a lecture wasn't offered at my home university which is the reason why I wanted to share the material with all students, who are curious about it, and translate the notes from Russian into English. Last but not least, the content is very interesting, in my opinion, and it is nice to have a small memory of the great time in Saint Petersburg. \\

The course is aimed for third-year undergraduate physics students who already have knowledge in classical mechanics, electrodynamics and quantum mechanics. One should be especially familiar with tensors and index notation, e.g. the Lagrangian description of electrodynamics is assumed to be known.
We will concentrate on the Hamiltonian method and the consequences of constraints. Some aspects might be missed, so there is no entitlement to completeness and this notes should be understood more like an introduction. \\

The first chapter contains some theory which is needed to analyse constrainted systems. In the following chapters, we use the knowledge to investigate some interesting examples and develop the theory further.
Sometimes, I changed the notation or the order to make it more readable and clear but I wanted to be as near to the lecture as possible to conserve the Russian style. In the end, there is a set of exercises for preparation for the exam.  \\

The lecture is based on the literature on the next page. Dirac was the first who realized the importance of this topic and developed the theory, so especially \cite{1} should be consultated for further reading and better explanations. \\

I would like to thank Ksusha for her beautiful and detailed handwritten notes and her help during the semester. \\


If you find any mistakes, please let me know. \\


January 03, 2019 \hfill Eugen Dizer 